\chapter{Ներածություն}

\begin{itemize}
  \item Տվյալների կառուցվածքների մասին,
  \item Ջավա լեզվի ընտրության մասին,
  \item Տիպերի մասին,
  \item Literate Programming֊ի մասին,
  \item \(O\)\textit{-մեծ} գրառման մասին։
\end{itemize}


\smallskip
Այս գրքի շարադրանքի համար ընտրել ենք \emph{Գրագետ ծրագրավորման}
(\emph{Literate Programming}, \cite{LitProgArt}, \cite{LitProgBook})
եղանակը: Այն մեզ հնարավորություն է տալիս տեքստի կոնկրետ հատվածում
շարադրանքի միտքը կենտրոնացնել առանցքային թեմայի վրա և հետաձգել կամ
տեղափոխել

\smallskip
Տվյալների կառուցվածքների հետ կատարվող գործողությունների՝ ալգորիթմների
բարդության արտահայտման համար կօգտագործենք \(O\)\textit{-մեծ} գրառումը։
%Ալգորիթմներին ու տվյալների կառուցվածքների նվիրված համարյա բոլոր գրքերում
%խոսվում է այս մեթոդի մասին։ Տես, օրինակ, \cite{a}, \cite{b} կամ \cite{c}։
