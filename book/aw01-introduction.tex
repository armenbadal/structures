
%
%
%
\chapter{Ներածություն}

Հարգելի ընթերցո՛ղ։

Ձեր առջև է մի գրքույկ, որը նախնական տեղեկություններ է պարունակում ինֆորմատիկայում և ծրագրավորման աշխարհում առավել հաճախ օգտագործվող ու քննարկվող տվյալների կառուցվածքների մասին։ Այս գրքույկը չի հավակնում մրցակցել այնպիսի դասագրքերի հետ, ինչպիսիք են, օրինակ, Նիկլաուս Վիրտի «Ալգորիթմներ և տվյալների կառուցվածքներ»֊ը \cite{nw-ads}, Ռոբերտ Սեջվիկի և ??? Ուեյնի «Ալգորիթմներ»֊ը \cite{rs-kw-al} կամ Ալֆրեդ Ահոյի, Ջոն Հոպկրոֆտի և Ջեֆրի Ուլմանի «Տվյալների կառուցվածքներ և ալգորիթմերը» \cite{ahu-dsa}։ Մեր նպատակն է եղել ինֆորմատիկա ուսումնասիրող I և II կուրսի ուսանողների համար պատրաստել մի «այբբենարան», որով նա կարող է ծանոթանալ «Տվյալների կառուցվածքներ» թեմային հիմունքներին և գիտելիքներ նախապատրաստել հետագա, ավելի մանրամասն ուսումնասիրությունների համար։ 

Ոչինչ այս գրքույկում նոր չէ և չի հորինվել կամ հայտնագործվել մեր կողմից։ 



