\chapter{Պարզ ու բաղադրյալ տիպեր}

\textit{հեռացնել կամ ձևափոխել}

Ծրագրավորման լեզուներում \emph{պարզ} (կամ \emph{ատոմար}) են կոչվում
այն տիպերը, որոնց արժեքները չեն տրոհվում ավելի մանր մասերի։ Ջավա
լեզվի պարզ տիպերն են. \texttt{boolean}, \texttt{byte}, \texttt{short},
\texttt{int}, \texttt{long}, \texttt{float}, \texttt{double} և
\texttt{char}:

\texttt{String} տիպը, թեև իր պարզ չէ, սակայն Ջավան հնարավորություն է
տալիս դրա արժեքների հետ աշխատել պարզ տիպերի նմանությամբ։

Բաղադրյալ տիպերի օբյեկտները կառուցվում են մի քանի նույնատիպ կամ
տարատիպ օբյեկտներից։ Նույնատիպ օբյեկտների հավաքածու է \emph{զանգվածը},
որը հիշողության անընդհատ տիրույթ զբաղեցնող միատեսակ օբյեկտների
հաջորդականություն է։ Օրինակ, հետևյալ հրամանը, \texttt{new}
գործողությամբ, ստեղծում է \texttt{byte} տիպի \(16\) տարրերի զանգված․

\begin{verbatim}
var bs = new byte[16];
\end{verbatim}

\noindent Զանգվածի տարրերն ինդեքսավորվում են \(0\)֊ից սկսվող ինդեքսներով և
հասանելի են ինդեքսավորման \texttt{[]} գործողության միջոցով։ Այսպես.

\begin{verbatim}
bs[0] = 3;      // առաջին տարրին վերագրել 3 արժեքը
bs[4] = bs[0];  // հինգերորդ տարրին վերագրել առաջին տարրի արժեքը
\end{verbatim}

Բացի միաչափ զանգվածը, որը կոչվում է \emph{վեկտոր}, հաճախ հանդիպում են
երկչափանի զանգվածներ՝ \emph{մատրիցներ}։ Օրինակ, հետևյալ հրամանը սահմանում
է իրական թվերի 4 տողեր և 6 սյուներ ունեցող մատրից.

\begin{verbatim}
var mx = new double[4][6];
\end{verbatim}

Ընդհանրապես, Ջավա լեզուն հնարավորություն է տալիս ստեղծել երկուսից մեծ
չաթողականություն ունեցող զանգվածներ ևս։

Իրարից տարբերվող բաղադրիչներով տիպերն արտահայտվում են դասերի (\texttt{class})
կամ գրառումների (\texttt{record}) միջոցով։ Օրինակ, շախմատի տախտակի վանդակը,
որը որոշվում է տառ-թիվ զույգով, կարելի է սահմանել հետևյալ կերպ.

\begin{verbatim}
public record Cell(char column, int row) {}
\end{verbatim}

Դասերի ու գրառումների նմուշները նույնպես ստեղծվում են \texttt{new} գործողությամբ։
Օրինակ, հետևյալ կոդը սահմանում և ստեղծում է շախմատի տախտակի մոդելը․

\begin{verbatim}
Cell[][] board = new Cell[8][8];
for(int r = 8; r > 0; --r)
    for( )
\end{verbatim}
