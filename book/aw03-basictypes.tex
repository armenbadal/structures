%
%
%
\chapter{Պարզ ու բաղադրյալ տիպեր}

Ծրագրավորման լեզուներում պարզ կամ ատոմար են կոչվում այն տիպերը, որոնց
արժեքները չեն տրոհվում ավելի մանր մասերի։ Օրինակ, \texttt{boolean} տիպի
\texttt{false} և \texttt{true} արժեքներն անտրոհելի են։ Նույն կերպ անտրոհելի
են նաև \texttt{int}, \texttt{double} կամ \texttt{byte} տիպերի արժեքները։

Բաղադրյալ տիպերի օբյեկտները կառուցվում են մի քանի նույնատիպ կամ տարատիպ
օբյեկտներից։ Նույնատիպ օբյեկտների հավաքածու է, օրինակ, զանգվածը, որը
հիշողության անընդհատ տիրույթ զբաղեցնող միատեսակ օբյեկտների հաջորդականություն
է։ Օրինակ.

\begin{verbatim}
byte[] bs = new byte[16];
\end{verbatim}

\noindent հրամանը սահմանում է 16 հատ \texttt{byte} տիպի օբյեկտների
զանգված։ Զանգվածի տարրերը հասանելի են ինդեքսավորման գործողության միջոցով։
Այսպես.

\begin{verbatim}
bs[0] = 3;      // առաջին տարրին վերագրել 3 արժեքը
bs[4] = bs[0];  // հինգերորդ տարրին վերագրել առաջին տարրի արժեքը
\end{verbatim}

Ինդեքսները սկսվում են 0-ից։ Եթե զանգվածն ունի 16 տարր, ապա առաջին տարրի
ինդեքսը 0-ն է, իսկ վերջին տարրինը՝ 15-ն է։

Բացի միաչափ զանգվածը, որը կոչվում է \emph{վեկտոր}, հաճախ հանդիպում են
երկչափանի զանգվածներ՝ \emph{մատրիցներ}։ Օրինակ, հետևյալ հրամանը սահմանում
է իրական թվերի 4 տողեր և 6 սյուներ ունեցող մատրից.

\begin{verbatim}
double[][] mx = new double[4][6];
\end{verbatim}

Զանգվածներn իրենց տարրերի հասանելիության գործողության հաստատուն բարդություն
են երաշխավորում անկախ տարրերի քանակից։

Իրարից տարբերվող բաղադրիչներով տիպերն արտահայտվում են դասերի (\texttt{class})
կամ գրառումների (\texttt{record}) միջոցով։ Օրինակ, շախմատի տախտակի վանդակը,
որը որոշվում է տառ-թիվ զույգով, կարելի է սահմանել հետևյալ կերպ.

\begin{verbatim}
public record Cell(char column, int row) {}
\end{verbatim}
