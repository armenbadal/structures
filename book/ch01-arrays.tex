\chapter{Զանգվածներ}

\emph{Զանգվածը}, թերևս, ծրագրավորման լեզուներում հանդիպող ամենապարզ
տվյալների կառուցվածքն է։ Այն բաղկացած է մեքենայի հիշողության մեջ
\emph{անընդհատ} հատված զբաղեցնող \emph{միատիպ} տարրերից։ Ջավա լեզուն
հնարավորություն է տալիս ստեղծել ու օգտագործել ինչպես միաչափ, այնպես
էլ երկու և ավելի չափողականություններ ունեցող զանգվածներ։ Հետևյալ
հրամանը հայտարարում է \texttt{char} տիպի տարրերի միաչափ զանգված․

\begin{verbatim}
char[] characters;
\end{verbatim}

Զանգվածի օբյեկտը ստեղծվում է \texttt{new} գործողությամբ՝ տիպից հետ
քառակուսի փակագծերում նաև տարրերի քանակը նշելով․

\begin{verbatim}
var characters = new char[32];
\end{verbatim}


Զանգվածների հիմնական առավելությունն այն է, որ դրանք թույլ են տալիս
իրենց տարրերին, \emph{ինդեքսի} միջոցով, դիմել հաստատուն՝ \(O(1)\),
բարդությամբ գործողություններով։

\section{Միաչափ զանգված}

Միաչափ զանգվածը կամ վեկտորը տվյալների կառուցվածք է, որի տարրերը
հաջորդաբար դասավորված են հիշողության անընդհատ տիրույթում։ Տարրերի
հասանելիության համար օգտագործվում է \(0\)֊ից սկսվող ինդեքսը։

Այստեղ կիրականացնենք \texttt{Array1D} դասը,

\nwfilename{ch01-array-impl.nw}\nwbegincode{1}\sublabel{NW2xbAX4-3DSt5d-1}\nwmargintag{{\nwtagstyle{}\subpageref{NW2xbAX4-3DSt5d-1}}}\moddef{Array1D դասը~{\nwtagstyle{}\subpageref{NW2xbAX4-3DSt5d-1}}}\endmoddef\nwnotused{Array1D\ դասը}
package structures;
public class Array1D<E extends Comparable> \{
    \LA{}Տարրերի զանգված~{\nwtagstyle{}\subpageref{NW2xbAX4-MMD3K-1}}\RA{}
    \LA{}Կոնստրուկտոր~{\nwtagstyle{}\subpageref{NW2xbAX4-0-1}}\RA{}
    \LA{}Զանգվածի տարրերի քանակը~{\nwtagstyle{}\subpageref{NW2xbAX4-11VMew-1}}\RA{}
    \LA{}Տարրերի հասանելիություն~{\nwtagstyle{}\subpageref{NW2xbAX4-2KcLCY-1}}\RA{}
    \LA{}Գծային որոնում~{\nwtagstyle{}\subpageref{NW2xbAX4-MMD3K.2-1}}\RA{}
    \LA{}Բինար որոնում~{\nwtagstyle{}\subpageref{NW2xbAX4-MMD3K.3-1}}\RA{}
\}
\nwendcode{}\nwbegindocs{2}\nwdocspar

\texttt{Array1D}-ի տարրերը պահելու համար սահմանենք \texttt{elements} 
հասարակ զանգվածը․

\nwenddocs{}\nwbegincode{3}\sublabel{NW2xbAX4-MMD3K-1}\nwmargintag{{\nwtagstyle{}\subpageref{NW2xbAX4-MMD3K-1}}}\moddef{Տարրերի զանգված~{\nwtagstyle{}\subpageref{NW2xbAX4-MMD3K-1}}}\endmoddef\nwused{\\{NW2xbAX4-3DSt5d-1}}
private final E[] elements;
\nwendcode{}\nwbegindocs{4}\nwdocspar

\nwenddocs{}\nwbegincode{5}\sublabel{NW2xbAX4-0-1}\nwmargintag{{\nwtagstyle{}\subpageref{NW2xbAX4-0-1}}}\moddef{Կոնստրուկտոր~{\nwtagstyle{}\subpageref{NW2xbAX4-0-1}}}\endmoddef\nwused{\\{NW2xbAX4-3DSt5d-1}}
public Array1D(int size) \{
    elements = (E[])new Object[size];
\}
\nwendcode{}\nwbegindocs{6}\nwdocspar

\nwenddocs{}\nwbegincode{7}\sublabel{NW2xbAX4-11VMew-1}\nwmargintag{{\nwtagstyle{}\subpageref{NW2xbAX4-11VMew-1}}}\moddef{Զանգվածի տարրերի քանակը~{\nwtagstyle{}\subpageref{NW2xbAX4-11VMew-1}}}\endmoddef\nwused{\\{NW2xbAX4-3DSt5d-1}}
public int size() \{
    return elements.length;
\}
\nwendcode{}\nwbegindocs{8}\nwdocspar

\nwenddocs{}\nwbegincode{9}\sublabel{NW2xbAX4-2KcLCY-1}\nwmargintag{{\nwtagstyle{}\subpageref{NW2xbAX4-2KcLCY-1}}}\moddef{Տարրերի հասանելիություն~{\nwtagstyle{}\subpageref{NW2xbAX4-2KcLCY-1}}}\endmoddef\nwalsodefined{\\{NW2xbAX4-2KcLCY-2}}\nwused{\\{NW2xbAX4-3DSt5d-1}}
public E get(int index) \{
    return elements[index];
\}
\nwendcode{}\nwbegindocs{10}\nwdocspar

\nwenddocs{}\nwbegincode{11}\sublabel{NW2xbAX4-2KcLCY-2}\nwmargintag{{\nwtagstyle{}\subpageref{NW2xbAX4-2KcLCY-2}}}\moddef{Տարրերի հասանելիություն~{\nwtagstyle{}\subpageref{NW2xbAX4-2KcLCY-1}}}\plusendmoddef
public void set(int index, E value) \{
    elements[index] = value;
\}
\nwendcode{}\nwbegindocs{12}\nwdocspar

\nwenddocs{}\nwbegincode{13}\sublabel{NW2xbAX4-MMD3K.2-1}\nwmargintag{{\nwtagstyle{}\subpageref{NW2xbAX4-MMD3K.2-1}}}\moddef{Գծային որոնում~{\nwtagstyle{}\subpageref{NW2xbAX4-MMD3K.2-1}}}\endmoddef\nwused{\\{NW2xbAX4-3DSt5d-1}}
public int search(E value) \{
    for( int i = 0; i < size(); ++i )
        if( get(i).equals(value) )
            return i;
    
    return -1;
\}
\nwendcode{}\nwbegindocs{14}\nwdocspar

\nwenddocs{}\nwbegincode{15}\sublabel{NW2xbAX4-MMD3K.3-1}\nwmargintag{{\nwtagstyle{}\subpageref{NW2xbAX4-MMD3K.3-1}}}\moddef{Բինար որոնում~{\nwtagstyle{}\subpageref{NW2xbAX4-MMD3K.3-1}}}\endmoddef\nwused{\\{NW2xbAX4-3DSt5d-1}}
public int binarySearch(E value) \{
    return binarySearchHelper(0, size()-1, value);
\}

private int binarySearchHelper(int l, int r, E v) \{
    int m = (l + r) / 2;
    if( m == v ) return 0;
\}
\nwendcode{}

\nwixlogsorted{c}{{Array1D դասը}{NW2xbAX4-3DSt5d-1}{\nwixd{NW2xbAX4-3DSt5d-1}}}%
\nwixlogsorted{c}{{Բինար որոնում}{NW2xbAX4-MMD3K.3-1}{\nwixu{NW2xbAX4-3DSt5d-1}\nwixd{NW2xbAX4-MMD3K.3-1}}}%
\nwixlogsorted{c}{{Գծային որոնում}{NW2xbAX4-MMD3K.2-1}{\nwixu{NW2xbAX4-3DSt5d-1}\nwixd{NW2xbAX4-MMD3K.2-1}}}%
\nwixlogsorted{c}{{Զանգվածի տարրերի քանակը}{NW2xbAX4-11VMew-1}{\nwixu{NW2xbAX4-3DSt5d-1}\nwixd{NW2xbAX4-11VMew-1}}}%
\nwixlogsorted{c}{{Կոնստրուկտոր}{NW2xbAX4-0-1}{\nwixu{NW2xbAX4-3DSt5d-1}\nwixd{NW2xbAX4-0-1}}}%
\nwixlogsorted{c}{{Տարրերի զանգված}{NW2xbAX4-MMD3K-1}{\nwixu{NW2xbAX4-3DSt5d-1}\nwixd{NW2xbAX4-MMD3K-1}}}%
\nwixlogsorted{c}{{Տարրերի հասանելիություն}{NW2xbAX4-2KcLCY-1}{\nwixu{NW2xbAX4-3DSt5d-1}\nwixd{NW2xbAX4-2KcLCY-1}\nwixd{NW2xbAX4-2KcLCY-2}}}%
\nwbegindocs{16}\nwdocspar

\nwenddocs{}

